\section{Bestehende Lösungen} \todo{Die anderen mehr Bashen!}
  Es gibt einige bestehende Ansätze, die die verschiedenen Aspekte des direktionalen Hörens auf eine technische Apparatur übertragen. Diesen begegnen wir in unserem alltäglichen Leben relativ häufig. Die einfachste Form eines solchen Verfahrens ist das Richtmikrofon. Dieses führt allerdings keine aktive Ortung durch, sondern kann lediglich in eine bestimmte Richtung besonders gut Schall aufnehmen. Um diese Fähigkeit für eine Richtungsbestimmung zu nutzen, müsste man also das Richtmikrofon drehen, oder es auf eine andere Weise aktiv nach der Schallquelle ausrichten. Ein weiterer Ansatz, dem wir in unserem alltäglichen Leben noch sehr viel häufiger begegnen, steckt in fast allen Mobiltelefonen. Diese filtern beim Telefonieren verschiedene Umgebungsgeräusche aus dem Mikrofonsignal, um die Sprachqualität zu verbessern. Die hierzu verwendeten Verfahren sind allerdings meist eher einfach gehalten und erlauben keine wirkliche Bestimmung der Herkunftsrichtung eines Geräusches. Man kann sich dieses Verfahren sehr gut als ein Richtmikrofon vorstellen, das einen gewissen Bereich hat, in dem es sehr empfindlich ist, wohingegen es in anderen Bereichen sehr unempfindlich ist. Bei dieser Technik passiert ein Teil der Geräuschunterdrückung in der Signalverarbeitung, also nach der eigentlichen Schallwandlung durch das Mikrofon. Hierdurch unterscheidet sich dieser Ansatz deutlich von dem des Richtmikrofones. Allerdings können auch mithilfe dieser Störgeräuschunterdrückung noch keine Positionen ermittelt werden. Auch moderne Hörgeräte verwenden ein ähnliches Verfahren, welches allerdings auch bei einer weiteren Distanz zwischen Mikrofon und Schallquelle funktioniert, und der gesuchten Lösung somit näher kommt. Sowohl die Geräuschunterdrückung in Handys, als auch die Filtertechniken in Hörgeräten verwenden meist zwei oder drei Mikrophone.
  \begin{figure}
    \centering
    \includegraphics[width=0.35\linewidth]{img/akusticCamera}
    \caption{Ein Beispiel für eine akustische Kamera \cite{camera}}
    \label{fig:camera}
  \end{figure}
  Eine andere existierende Lösung ist die akustische Kamera (siehe Abb. \ref{fig:camera}). Sie wird dazu verwendet, lärm-emittierende Positionen an Produkten zu finden, um diese optimieren zu können \cite{camera}. Die akustische Kamera verwendet allerdings bedeutend mehr Mikrofone als die anderen Verfahren. Einige Modelle verwenden mehr als 350 Mikrofone \cite{nmics}.\todo{warum ist akust. Kamera nicht die Lösung: Ist auf best. Produkte gemünzt} Da uns alle vorhandenen Lösungen nicht zufrieden gestellt haben, wollten wir ein eigenes Verfahren für die akustischen Richtungsbestimmung entwickeln, das schon mit einer geringen Anzahl von Mikrofonen eine komplette Richtungsbestimmung ermöglicht und leicht erweiterbar auf weitere Auswertungsschritte ist.
