\section{Ausblick} \todo{rewrite; sachen die gehen in Arbeit packen; Amplituden komplett raus}
\addcontentsline{toc}{section}{Ausblick}
\iffalse
Da unser Verfahren über die Least Squares Methode sehr leicht auf eine beliebige Anzahl von Mikrofonen erweiterbar ist, wollen wir die Genauigkeit und insbesondere die Reichweite der Richtungsbestimmung noch weiter erhöhen, indem wir anstelle von vier Mikrofonen acht Mikrofone verwenden. Dies würde es uns außerdem erlauben, die Richtungsbestimmung unabhängig von der Schallgeschwindigkeit  durchzuführen, indem diese als eine weitere Variable eingeführt wird. Damit würde sich die Genauigkeit der Richtungsbestimmung weiter verbessern.

Weiterhin könnte die Richtungsbestimmung verbessert werden, indem auch die Amplituden mit einbezogen werden. Diese stehen durch die Fourier-Transformation bereits zur Verfügung und könnten ebenfalls zu einer weiteren Verbesserung der Reichweite und Genauigkeit der Richtungsbestimmung beitragen.
\fi
\todo{vll. mal wirklich unabhöngikeit v. schallgeschwindigkeit einbauen?}
Dadurch, dass wir anstatt einer Richtung eigentlich eine Position bestimmen, könnte man unser Verfahren nach weiterer Optimierung auch für die Positionsbestimmung von Schallquellen verwenden.
