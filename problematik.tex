\section{Problematik}
\todo{Jaro: teilweise neuer text}
Der Mensch hat die Fähigkeit, direktional zu hören, also die Richtung zu bestimmen aus der ein Geräusch kommt. Dies bringt ihm enorme Vorteile bei der Erkennung von gesprochener Sprache und bei anderen akustischen Aufgaben. Diese Fähigkeit auf eine technische Apparatur zu übertragen und die Vorteile des räumlichen Hörens auch für diese nutzbar zu machen, hätte viele Anwendungsgebiete, die unser alltägliches Leben erleichtern könnten.\\
Ein gutes Beispiel für eine Solche Anwendung wäre ein Rettungsroboter, der Hilfesuchende Menschen anhand von Hilferufen lokalisiert. Auch könnte man verschiedene Hilfsmittel für den Menschen Konstruieren, die ein solches technisches Verfahren verwenden. Ein Beispiel hierfür wäre ein Halm, den sich Feuerwehrleute aufsetzen können um in verrauchten Umgebungen Geräuschen wie schreienden Babys zu folgen.\\
Jedoch sind auch Anwendungen aus komplett anderen Anwendungsbereichen denkbar: So wäre es möglich eine Anwendung zu entwickeln, die die Richtungsinformationen verwendet, um das Audiosignal zu filtern. Hiervon würde zum Beispiel die Spracherkennung profitieren, da sie sehr störungsfreie Audiosignale benötigt und immer nur ein Sprecher von Interesse ist\cite{Spracherkennung}. Auch andere Anwendungen, die störungsfreie Audiosignale Benötigen könnten die Vorteile der Richtungserkennung nutzen. 
