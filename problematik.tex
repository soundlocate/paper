\section{Akustische Richtungsbestimmung} Der Mensch hat die Fähigkeit, direktional zu hören, also die Richtung zu bestimmen, aus der ein Geräusch kommt. Dies bringt ihm enorme Vorteile bei der Erkennung von gesprochener Sprache und bei anderen akustischen Aufgaben. Diese Fähigkeit auf eine technische Apparatur zu übertragen und die Vorteile des räumlichen Hörens auch für diese nutzbar zu machen, hätte viele Anwendungsgebiete, die unser alltägliches Leben erleichtern könnten.\\
Ein gutes Beispiel für eine solche Anwendung wäre ein Rettungsroboter, der hilfesuchende Menschen anhand von Hilferufen lokalisiert. Auch könnte man verschiedene Hilfsmittel für den Menschen konstruieren, die helfen Personen in verrauchten Umgebungen zu finden.
Weiterhin sind auch Anwendungen aus komplett anderen Anwendungsbereichen denkbar: So wäre es möglich eine Anwendung zu entwickeln, die die Richtungsinformationen verwendet, um Audiosignale zu filtern. Dadurch könnte man Anwendungen die störungsfreie Audiosignale benötigen, wie zum Bespiel Spracherkennung\cite{Spracherkennung}, verbessern.
