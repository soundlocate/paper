\section{Konzept}
Bei der Entwicklung unseres neuen Verfahrens zur Richtungsbestimmung haben wir uns stark am menschlichen Gehör und seiner Fähigkeit des Richtungshörens orientiert. Da beim menschlichen Gehör verschiedene Verarbeitungssstufen durchlaufen werden, ist auch unsere Software modular aufgebaut (siehe Tabelle~\ref{analog}). Nachfolgend werden die einzelnen Module in der Reihenfolge, in der sie das Signal durchläuft, beschrieben.\\
Um unser Konzept umzusetzen und zu evaluieren, haben wir uns entschlossen, einerseits eine Computersimulation zu schreiben und andererseits mithilfe einer realen Messapparatur zu überprüfen, ob unser Verfahren auch in der Realwelt verwendbar ist. Der reale Aufbau wurde zuerst in einer Schallkammer, also einem Raum mit wenigen akustischen Störquellen von außen und mit wenigen akustischen Reflexionen an den Wänden, und erst danach unter Einfluss von Störungen getestet. Dieses Vorgehen hat den Vorteil, dass man zuerst die Theorie entwickeln kann, und danach die Theorie und ihre Implementation in den verschiedenen Schritten zum Endprodukt immer weiter an Effekte aus der realen Welt, wie zum Beispiel Rauschen, anpassen kann. Auch hierbei hilft der modulare Aufbau, da so möglichst viele Programmteile sowohl in der Simulation als auch in realen Aufbauten verwendet werden können.\\
Unser Konzept sieht vier Module vor, welche über TCP/IP und Websockets verbunden sind. Dies sind Standards, mit denen über ein Computernetzwerk Daten ausgetauscht werden können~\cite{tcp}~\cite{websockets}. Sie garantieren, dass alle Daten in der Reihenfolge, in der sie losgeschickt werden, ankommen. Die Verwendung von Netzwerkprotokollen hat den Vorteil, dass die Module nicht unbedingt auf demselben Computer ausgeführt werden müssen und so der Rechenaufwand bei Bedarf auf mehrere Computer verteilt werden kann.
\begin{table}[h]
	\centering
	\begin{tabular}{ll}
      Mensch            & Unser Verfahren                                   \\ \midrule
      Ohren             & Mikrofonarray                              \\
      Haarzellen, Basilarmembran im Ohr & Fourier Transformation                     \\
      Richtungsbestimmung im Gehirn            & Computeralgorithmus zur Richtungsbestimmung                       \\
      Selektives Hören  & Weiterverarbeitung durch Programme Dritter
	\end{tabular}
    \caption{Die Analogie zwischen dem menschlichen Gehör und unserem Verfahren. (Erläuterungen siehe~\ref{sec:module1} bis~\ref{sec:module4})\label{analog}}
\end{table}

\subsection{Modul 1: Eingabe/Aufnahme\label{sec:module1}}
Das erste Modul in dieser Kette stellt die Rohdaten für die weitere Verarbeitung bereit. Diese können von echten Mirkofonen stammen oder aus einer Simulation. Dieses Modul entspricht den menschlichen Ohren und ihrer Aufgabe, den Schall aufzunehmen.

\subsection{Modul 2: Signal nach Frequenzen aufteilen}
Im zweiten Modul der Kette wird das Signal, welches aus dem ersten Modul stammt, in die einzelnen im Signal enthaltenen Wellen aufgeteilt, also von einer zeitaufgelösten Form in ein frequenzaufgelöstes Signal umgewandelt. Nach diesem Schritt liegt also für jede im Signal vorkommende Frequenz eine Amplitude und eine Phase vor. Dieser Schritt wird im menschlichen Ohr durch eine mechanische Konstruktion, die verschiedene Haarzellen für verschiedene Frequenzen anregt, vorgenommen. Unser technisches Verfahren verwendet hierzu die Fouriertransformation. Diese Transformation hat den Vorteil, dass jede Frequenz mit ihrer zugehörigen Amplitude und Phase einzeln analysiert werden kann.
In diesem Modul werden außerdem Frequenzen, welche nicht oder nur sehr leise in dem Eingangssignal vorkommen, herausgefiltert, um das Rauschen und den benötigten Rechenaufwand in den nächsten Modulen zu verringern.

\subsection{Modul 3: Richtungsbestimmung}
Das dritte Modul empfängt die Daten des vorherigen Moduls und errechnet zunächst für jede Frequenz aus den transformierten Daten den Gangunterschied zwischen den Signalen der Mikrofone. Aus diesen werden dann die möglichen Ursprungsrichtungen der einzelnen Sinusschwingungen ermittelt. Da dieser Vorgang für jede Frequenz einzeln vorgenommen wird, können auch mehrere Schallquellen mit unterschiedlichen Frequenzen gleichzeitig untersucht werden. Im menschlichen Gehör wird diese Richtungsbestimmung durch Prozesse im Gehirn, die Intensitäts- und Laufzeitdifferenzen ausnutzen, durchgeführt.

\subsection{Modul 4: Ausgabe\label{sec:module4}}
Das letzte Modul in der Signalkette ist die Ausgabe. Sie bekommt die fertig gruppierten Messergebnisse über eine Netzwerkverbindung und bereitet sie für den Nutzer auf. Dieses Modul kann auf die jeweilige Anwendung angepasst werden und es ist möglich mehrere Output-Module mit dem gleichen Richtungsmodul verbunden werden. Auch Endanwendungen, die das Richtungshören verwenden wollen, sind Ausgabemodule.\\
Der Mensch kann sich zum Beispiel mithilfe der bestimmten Richtungen auf einzelne Schallquellen konzentrieren und andere Geräusche ausblenden.